\documentclass[]{article}

\usepackage{listings}

%opening
\title{CSE150 -- Project 3\\ Networks and Distributed Systems}
\author{Group Thimbles}

\begin{document}

\maketitle


\section{Networking syscalls}
\subsection{connect()}

\paragraph{Port mapping}
Ports are mapped to sockets for both incoming and outgoing connections.\\




\noindent Pseudocode:

 Attempt to initiate a new connection to the specified port on the specified
 remote host, and return a new file descriptor referring to the connection.
 connect() does not give up if the remote host does not respond immediately.
 
 Returns the new file descriptor, or -1 if an error occurred.

\begin{lstlisting}

int connect(host, port){
	disable interrupts
	create a new socket s and assign it to a free port in (0,127)
	s.state = SYN_SENT
	send SYN packet
	block until SYN/ACK recv'd // timeout breaks this
	s.state = ESTABLISHED
	enable interrupts
	return s.fileDescriptor
}
\end{lstlisting}

\subsection{accept()}

Pseudocode:

\begin{lstlisting}
int acccept(port){
	disable interrupts
	if there are connections waiting on port
		create a new socket s, assign it that port
	else return -1
	s.state = ESTABLISHED
	send SYN/ACK
	enable interrupts
	return s.fileDescriptor
}
\end{lstlisting}

\subsection{write()}

write() allows the connection to write to the network. Attempts to write a buffer of bytes to
the socket. If the socket is not Established it returns 1.
\begin{lstlisting}
int write(fileDescriptor, buffer, count){
	
	if(state == ESTABLISHED && offset + length <= buf.length))
	{
		new packet
		int bytePos = offset;
		int endPos = offset + length;
		while(bytePos < endPos){
		
		System.arraycopy(buf, bytePos, contents, 8, amountSend);
		}
	}
	
	...
	<netcode>
	...
}
\end{lstlisting}

\subsection{read()}
read() allows the connection to read from the network. Attempts to read a number of bytes from the socket. If it is closed and there are no bytes in the buffer, it will return 1, otherwise return the number of bytes read. It does not block.
\begin{lstlisting}
int read(fileDescriptor, buffer, count){
	...
	// for a socket
	if (s.isOpen){
		read count bytes
		return bytes successfully read
	} else {
		if (socket isn't empty){
			read count bytes
			if (socket is empty)
				delete socket		
			return bytes successfully read	
		}	
	}
}


\end{lstlisting}

\section{Threads}



\subsection{Send thread}

\subsection{Receive thread}
\subsection{Timeout thread}
This thread works like waitUntil, where it loops through the existing sockets and checks for any that have lived past their timeout value. If they have, it closes that socket.

\section{Test cases}
\subsection{connect()}
\begin{itemize}
	\item Attempt to open a connection to a node that doesn't exist
	\subitem Check that connect() blocks
	\item Open a connection to an existing node
	\subitem Check that connect() returns
	\item Close an already-open connection
	\subitem Verify that socket is closed on both sides
	\item Open multiple connections to the same receiving port
	\subitem Check that they all send/receive data
	\item Open a connection, close it and re-open it
	\item packet drop during connect()
\end{itemize}
\subsection{accept()}
\begin{itemize}
	\item Accept a waiting connection
	\item Accept multiple waiting connections on the same port
	\item Accept multiple waiting connections to different ports
	\item Return from accept() on a port that doesn't have a connection waiting
	\item packet drop during accept()
\end{itemize}
\subsection{close()}
\begin{itemize}
	\item Close a connection that doesn't exist
	\item Close a connection that exists
	\subitem Check that it's actually closed
	\item Close a connection twice in a row
\end{itemize}
\subsection{read/write()}
\begin{itemize}
	\item simple read write.
	\item read returns -1 during remote host disconnection 
	\item read returns even after disconnect
	\item write returns -1 during remote host disconnect
	\item write flushes data after close
\end{itemize}

\subsection{title}

\end{document}
